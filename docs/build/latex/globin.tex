%% Generated by Sphinx.
\def\sphinxdocclass{report}
\documentclass[letterpaper,10pt,english]{sphinxmanual}
\ifdefined\pdfpxdimen
   \let\sphinxpxdimen\pdfpxdimen\else\newdimen\sphinxpxdimen
\fi \sphinxpxdimen=.75bp\relax

\PassOptionsToPackage{warn}{textcomp}
\usepackage[utf8]{inputenc}
\ifdefined\DeclareUnicodeCharacter
% support both utf8 and utf8x syntaxes
  \ifdefined\DeclareUnicodeCharacterAsOptional
    \def\sphinxDUC#1{\DeclareUnicodeCharacter{"#1}}
  \else
    \let\sphinxDUC\DeclareUnicodeCharacter
  \fi
  \sphinxDUC{00A0}{\nobreakspace}
  \sphinxDUC{2500}{\sphinxunichar{2500}}
  \sphinxDUC{2502}{\sphinxunichar{2502}}
  \sphinxDUC{2514}{\sphinxunichar{2514}}
  \sphinxDUC{251C}{\sphinxunichar{251C}}
  \sphinxDUC{2572}{\textbackslash}
\fi
\usepackage{cmap}
\usepackage[T1]{fontenc}
\usepackage{amsmath,amssymb,amstext}
\usepackage{babel}



\usepackage{times}
\expandafter\ifx\csname T@LGR\endcsname\relax
\else
% LGR was declared as font encoding
  \substitutefont{LGR}{\rmdefault}{cmr}
  \substitutefont{LGR}{\sfdefault}{cmss}
  \substitutefont{LGR}{\ttdefault}{cmtt}
\fi
\expandafter\ifx\csname T@X2\endcsname\relax
  \expandafter\ifx\csname T@T2A\endcsname\relax
  \else
  % T2A was declared as font encoding
    \substitutefont{T2A}{\rmdefault}{cmr}
    \substitutefont{T2A}{\sfdefault}{cmss}
    \substitutefont{T2A}{\ttdefault}{cmtt}
  \fi
\else
% X2 was declared as font encoding
  \substitutefont{X2}{\rmdefault}{cmr}
  \substitutefont{X2}{\sfdefault}{cmss}
  \substitutefont{X2}{\ttdefault}{cmtt}
\fi


\usepackage[Bjarne]{fncychap}
\usepackage{sphinx}

\fvset{fontsize=\small}
\usepackage{geometry}


% Include hyperref last.
\usepackage{hyperref}
% Fix anchor placement for figures with captions.
\usepackage{hypcap}% it must be loaded after hyperref.
% Set up styles of URL: it should be placed after hyperref.
\urlstyle{same}

\addto\captionsenglish{\renewcommand{\contentsname}{Contents:}}

\usepackage{sphinxmessages}
\setcounter{tocdepth}{1}



\title{globin Documentation}
\date{Feb 11, 2021}
\release{0.1}
\author{Dusan Vukadinovic}
\newcommand{\sphinxlogo}{\vbox{}}
\renewcommand{\releasename}{Release}
\makeindex
\begin{document}

\pagestyle{empty}
\sphinxmaketitle
\pagestyle{plain}
\sphinxtableofcontents
\pagestyle{normal}
\phantomsection\label{\detokenize{index::doc}}


\sphinxstylestrong{globin} is a Python package for spectropolarimetric inversion wrapped around \sphinxhref{https://www2.hao.ucar.edu/spectropolarimetry/rh}{RH code} . Main purpose of this package is to infer atomic line parameters (oscilator strength and rest wavelength) alongside
the atmospheric parameters. For the detailed description of solving radiative transfer equation, read RH documentation.


\chapter{Introduction}
\label{\detokenize{user/introduction:introduction}}\label{\detokenize{user/introduction:intro}}\label{\detokenize{user/introduction::doc}}
Main purpose of \sphinxstylestrong{globin} package is to invert high\sphinxhyphen{}resolution NUV solar observations from SUSI instrument onboard SUNRISE III balloon\sphinxhyphen{}borne mission for inference of atomic line parameters. Package is able to invert ocilator strength (log(gf)) and rest wavelength of spectral line along with atmospheric parameters (temperature, vertical velocity, micro\sphinxhyphen{}turbulent velocity, magnetic field strength, field inclination and azimuth).

Parametrization of atmospheric structure is done using node approach where in node we have value of given physical parameters. Code minimizes difference between provided observed profiles and synthesised spectra using Levenberg\sphinxhyphen{}Marquardt algorithm.

In case of atomic parameters we have implemented a global minimization. In this approach, atomic parameters are called global and are the same for each observed Stokes profile in given field\sphinxhyphen{}of\sphinxhyphen{}view. This way, we have coupled information from all the pixels to have better inference of atomic parameters.

Atmospheric structure is assumed to be given in optical depth scale which is then finly interpolated on scale \sphinxhyphen{}6 to 1 with step size of 0.1. Interpolation between atmospheric nodes is performed with Bezier’s 2nd and 3rd order polynomials.

Response function, necessary for spectropolarimetric inversion using LM algorithm, are calculated numerical using \sphinxcode{\sphinxupquote{rf\_ray}} executable of RH. It is modified executable of \sphinxcode{\sphinxupquote{sovleray}}.


\section{Quick start}
\label{\detokenize{user/introduction:quick-start}}

\section{Atmosphere}
\label{\detokenize{user/introduction:atmosphere}}
Input atmosphere


\section{Synthesis}
\label{\detokenize{user/introduction:synthesis}}

\section{Inversion}
\label{\detokenize{user/introduction:inversion}}

\chapter{Installation}
\label{\detokenize{user/installation:installation}}\label{\detokenize{user/installation:install}}\label{\detokenize{user/installation::doc}}
To have an updated version of \sphinxstylestrong{globin} package, download it from \sphinxhref{https://gitlab.gwdg.de/dusan.vukadinovic01/atoms\_invert}{gitlab repo}. To install the \sphinxstylestrong{globin} package type:

\begin{sphinxVerbatim}[commandchars=\\\{\}]
pip3 install \PYGZhy{}e /path/to/package
\end{sphinxVerbatim}

Since the package is wrapped around RH it requieres also and functional installation of the code which can be downloaded from \sphinxcode{\sphinxupquote{here}}.


\section{Test}
\label{\detokenize{user/installation:test}}
To test the package, copy the sample files located in ‘globin/test’ directory and simply run \sphinxcode{\sphinxupquote{python run.py}}. In the terminal, it will write down the current progress of the test. Test results will then be writen inside \sphinxtitleref{result} directory.


\section{Dependencies}
\label{\detokenize{user/installation:dependencies}}
Package is written for Python 3.6+ interpreter and it is depended on following packages:

subprocess\textgreater{}=

multiprocessing\textgreater{}=

astropy\textgreater{}=

os\textgreater{}=

sys\textgreater{}=

time\textgreater{}=

copy\textgreater{}=

numpy\textgreater{}=

matplotlib\textgreater{}=

time\textgreater{}=

rh
\begin{quote}

io

xdrlib
\end{quote}



\renewcommand{\indexname}{Index}
\printindex
\end{document}